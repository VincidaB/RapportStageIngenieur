\litmus, qui signifie \textit{Linux Testbed for Multiprocessor Scheduling in Real-Time Systems} est un moyen de développer des applications temps réel sur le noyau Linux. Il contient des modifications au noyau habituel de Linux, des interfaces utilisateurs permettant d’interagir a bas niveau avec l’ordonnancement destâches sous Linux, ainsi qu'une infrastructure de traçage de l'exécution de l’ordonnanceur.
\litmus à été développé par Björn B Brandenburg \cite{brandenburg2011scheduling} afin de faciliter la recherche et la comparaison des algorithmes d'ordonnancement. Actuellement, beaucoup de publications utilisent \litmus afin de comparer différents protocoles de gestion de resources partagées par plusieurs processeurs. Mais \litmus est aussi utilisé pour sa facilité à être implémenté sur des plateformes récente dù au fait qu'il est construit par dessus le noyau Linux et que ce dernier est le système d'exploitation qui supporte le plus de plateformes. 

Ce dernier point est principalement pourquoi nous avons choisis \litmus comme système d'exploitation sur lequel nous implémenterons des algorithmes d’ordonnancement pour la carte de développement ROCK960. Les autres candidats, comme FreeRTOS, étaient souvent dirigés vers les microcontrôleurs ou bien n'étaient simplement pas compatibles avec la carte de développement. 