Comme le montre le listing de code \ref{annexe:p-edf}, je me suis appuyé sur la librairie \texttt{litmus/edf\_common.h} fournie dans \litmus. J'ai donc ensuite décidé d'implémenter un algorithme d'ordonnancement qui ne l'utilisait pas. J'ai donc choisi d'implémenter un algorithme RM (Rate Monotonic) partitionné. Cet algorithme est plus simple que P-EDF car il ne prend pas en compte les échéances des taches. Il suffit alors de trier les taches par période croissante et de les ordonnancer en fonction de leur période. Cependant, cet algorithme ne permet pas de garantir l'ordonnançabilité des taches. En effet, il existe des ensembles de taches qui ne sont pas ordonnançables par cet algorithme alors qu'ils le sont par P-EDF. Cependant, cet algorithme est plus simple à implémenter et permet de se familiariser avec les fonctions de \litmus.


\subsubsection{Implémentation}

Pour implémenter un algorithme RM partitionné, j'ai du réimplémenté les fonctions de \\ \texttt{litmus/edf\_common.h} pour suivre l'ordonnancement RM. Notre algorithme P-EDF faisait appel aux fonctions :
\begin{itemize}
    \item \texttt{edf\_domain\_init}, qui initialise le domaine temps réel avec l'ordre qui régit la priorité des tâches
    \item \texttt{edf\_preemption\_needed}, qui vérifie si la tâche en cours d'exécutions doit être préemptée
\end{itemize}
J'ai donc implémenté les fonctions :
\begin{itemize}
    \item \texttt{rm\_domain\_init}
    \item \texttt{rm\_preemption\_needed}
\end{itemize}

\begin{lstlisting}[style=cstyle, caption={Fonction \texttt{rm\_domain\_init}}, label={annexe:rm_domain_init}]
void rm_domain_init(rt_domain_t* rt, check_resched_needed_t resched,
					release_jobs_t release)
{
	rt_domain_init(rt, rm_ready_order, resched, release);
}
\end{lstlisting}

La complexitude de cette fonction se cache derière la nouvelle fonction d'ordre implémenté sous le nom de \texttt{rm\_ready\_order}. Cette fonction est passé en paramètre à \texttt{rt\_domain\_init} et permet d'initialiser le domaine temps réel avec l'ordre qui régit la priorité des tâches sous RM. Comme le montre le listing de code \ref{annexe:rm_common}, cette fonction est simple et permet de trier les tâches par période croissante. En cas, d'égalité, la priorité est départagée par PID croissant (la tâche avec le PID le plus petit est la plus prioritaire). On effectue aussi d'autres vérifications sur les tâches comme leur nature (tâche temps réel ou non), si deux fois la même tâche est passée en argument, ou encore si une des tâche est \texttt{NULL} (cas ou qu'une seul tâche n'est présente).

\subsubsection{Résultats et essais}

En raison de la grande quantité de nouveau code, faire fonctionner cet algorithme à nécessité une plus grande phase de débogage. J'ai donc utilisé l'outil \textit{feather-trace} pour tracer l'ordonnancement réel de cet algorithme. De cela, j'ai pu déterminer les étapes qui ne fonctionnaient pas. Par exemple, voici un exemple d'un essai avec plusieurs problèmes : 
\begin{figure}[H]
    \centering
    \includegraphics[width=0.75\paperwidth]{Images/schedule_host=rock960_scheduler=DEMO_trace=notstoped.png}
    \caption{Exemple d'ordonnancement avec défauts}
\end{figure}

Premièrement, l’ordonnanceur ne stipule pas a \textit{feather-trace} la fin d'exécution d'une tâches. Cela peut être corrigé en  appelant \texttt{sched\_trace\_task\_completion(prev, budget\_exhausted);} lors de la fin d'un job.
Secondement, on peut voir que la deuxième tâche se fait préempter en $t=5ms$ par la première tâche, plus prioritaire. Cependant, l'exécution de cette deuxième tâche ne continue pas lorsque le processeur est libre. Cet erreur était alors du à une erreur de logique dans la fonctions principale d’ordonnancement dans laquelle je ne replaçait pas les taches préemptés dans la \texttt{ready\_queue}.

Après ces erreurs corrigées voici la résulatat de l'ordonancement de \color{red}XXX \color{black} tâches par RM toutes deux lancées sur le même processeur:


\begin{center}
    \color{red} AJOUTER IMAGE DE L'ORDONANCEMENT DE RM
\end{center}

On peut aussi lancer les taches avec un \textit{offset} afin de voir si l'ordonnancement est correcte. Ce décalage est donné en tant que paramètre a la commande \texttt{rt-spin} de \textit{liblitmus} (paramètre -o). Voici un exemple d'ordonnancement avec un \textit{offset} de XXms pour la deuxième tâche.
\begin{center}
    \color{red} AJOUTER IMAGE DE L'ORDONANCEMENT DE RM AVEC OFFSET
\end{center}

On peut aussi lancer un plus grand nombre de tâches sur une multitude de processeurs (la carte de développement en ayant 6), et on obtient le tracé de tâches suivantes : 
\begin{center}
    \color{red} AJOUTER IMAGE DE L'ORDONANCEMENT DE RM PLEIN DE TACHES
\end{center}


Enfin, on peut aussi voir ce qu'il se passe lorsque l'on cherche a ordonnancer les même tâches que celles que l'on à ordonnancé avec P-EDF et que l'on peut voir dans la figure \ref{fig:edf-schedualibility-demo}.

\begin{center}
    \color{red}IL FAUT QUE JE FASSE L'ESSAI, MAIS NORMALEMENT C'EST PAS ORDONNANCABLE
\end{center}

Cela montre bien que RM est moins performant que P-EDF car il ne permet pas d'ordonnancer toutes les tâches ordonnançables par P-EDF. Cependant, il est plus simple à implémenter et m'a permis de me familiariser avec les fonctions de \litmus.


