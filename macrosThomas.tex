\newcommand{\tset}{\Gamma} % task set
\newcommand{\confer}{\textit{cf. }}
\newcommand{\trate}[2]{r_{#1,#2}} % r_{i,j}$  task i  processor j
\newcommand{\rate}[1]{r_{#1}}
\newcommand{\proc}[1]{\pi_{#1}}  %processor
\newcommand{\pset}{\Pi} % processor set
\newcommand{\load}[1]{C_{#1}}
\newcommand{\deadline}[1]{D_{#1}}
\newcommand{\offset}[1]{O_{#1}}
\newcommand{\U}[1][]{%
  \ifx\relax#1\relax
    U% Si la macro est appelée sans paramètre, affiche simplement U
  \else
    U(#1)% Si la macro est appelée avec un paramètre, affiche U(param)
  \fi
}
\newcommand{\period}[1]{T_{#1}}
\newcommand{\refractoryperiod}[1]{T_{#1}}
\newcommand{\figref}[1]{\figurename~\ref{#1}}
\newcommand{\arrref}[1]{Table~\ref{#1}}
\newcommand{\secref}[1]{Section~\ref{#1}}
\newcommand{\defeq}{\doteq}
\DeclareMathOperator{\lcm}{lcm}

% Generic task command

%Use : \task[(<key>=<val>)?(,(<key>=<val>)*]{<name>}
% with keys :
%   -symbol for the task symbol                 optional    (default : \tau)
%   -phase for the task phase                   optional    (default : {})
%   -iteration for the task iteration number    optional    (default : {})
%   -name : task's name                          mandatory


% Keys for parsing, in "mt" spacename (for mytask)
\makeatletter
\define@key{mytask}{symbol}{\def\mt@symbol{#1}}
\define@key{mytask}{phase}{\def\mt@phase{#1}}
\define@key{mytask}{iteration}{\def\mt@iteration{#1}}

% Default values
\setkeys{mytask}{symbol=\tau, iteration=\mbox{}, phase={}}

\newcommand{\task}[2][]{
\begingroup%
\setkeys{mytask}{#1}% parsing
\ifx\mt@phase\empty % printing (or not) the comma if a phase is givent
\mt@symbol_{#2}^{\mt@iteration}
\else
\mt@symbol_{#2,\mt@phase}^{\mt@iteration}
\fi
\endgroup
}
\makeatother

\makeatletter
\define@key{wcet}{phase}{\def\wcet@phase{#1}}

\setkeys{wcet}{phase={}}

\newcommand{\wcet}[2][]{
\begingroup
\setkeys{wcet}{#1}
\ifx\wcet@phase\empty
C_{#2}
\else
c_{#2}^{\wcet@phase}
\fi
\endgroup
}
\makeatother

\newcommand{\job}[2]{j_{#1}^{#2}}% job
\newcommand{\entiers}[2]{\left\llbracket#1,#2\right\rrbracket}
\newcommand{\set}[1]{\left\{#1\right\}}
\DeclareMathOperator{\gcf}{gcf}