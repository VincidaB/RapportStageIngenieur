\subsubsection{Installation d'une image précompilée}

Pour premier tester l'Installation de linux sur la carte de développement, j'ai utilisé une image de la distribution Ubuntu fournie par le fabriquant 96Boards disponible sur leur site. Cette image se présente sous la forme d'une archive au format \texttt{.tar.gz}. Elle contient à la fois le \gls{bootloader}, le noyau Linux, et le système de fichier. Cette image (\texttt{system.img}) peut alors être gravée (ou \textit{flashée}) sur une carte micro SD. 

Depuis un terminal, en se déplaçant dans le dossier de l'archive extraite, on exécute la commande suivante : 
\begin{lstlisting}[style=command]
$ sudo dd if=system.img of=/dev/XXX bs=4M oflag=sync status=noxfer
\end{lstlisting}

\begin{center}
    \color{red}
    EXPLIQUER CE QUE FAIT CETTE COMMANDE \\
    Aussi dire en quoi on s'en servira dans des scripts afin d'accélérer le développement.
\end{center}


\subsubsection{Compilation de Linux depuis le code source}\label{sec:compilation-linux-source}

Afin d'utiliser une version de Linux différente de la version précompilé par le fabriquant de la carte de développement, il faut se premièrement se procurer le code source du noyau Linux. Celui-ci est disponible sur un dépôt de code \gls{git} hébergé par GitHub. Il est disponible a l'adresse \texttt{https://github.com/torvalds/linux}, sous le profile du créateur de Linux : Linus Torvalds.


Ou est le code source linux ? Expliqué que j'étais déjà familié avec git et github pour des projets persos et tout.

Comment changer de version du code cloné. 

comment modifier quels modules sont chargés lors de la compilation de linux, interface graphique <-> fichier de config.
\subsubsection{Compilation croisée}

Toolchain : qu'est ce que c'est, de quoi elle est constituée ? Expliquer que l'on compile sur du x86 mais qu'on veut compiler pour du ARMv8xxx.

Variables d'environement? Qu'est ce que c'est sous linux, comparer a des 

Réalisation de scripts linux pour accélérer le développement. Que doit on charger pour charger le nouveau code compilé ?

Copy de l'image du noyau pour faire encore plus rapide.

