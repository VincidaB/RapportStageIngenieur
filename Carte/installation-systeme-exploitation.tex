\subsubsection{Installation d'une image précompilée}

Pour premier tester l'Installation de linux sur la carte de développement, j'ai utilisé une image de la distribution Ubuntu fournie par le fabriquant 96Boards disponible sur leur site. Cette image se présente sous la forme d'une archive au format \texttt{.tar.gz}. Elle contient à la fois le \gls{bootloader}, le noyau Linux, et le système de fichier. Cette image (\texttt{system.img}) peut alors être gravée (ou \textit{flashée}) sur une carte micro SD. 

Depuis un terminal, en se déplaçant dans le dossier de l'archive extraite, on exécute la commande suivante : 
\begin{lstlisting}[style=command]
$ sudo dd if=system.img of=/dev/XXX bs=4M oflag=sync status=noxfer
\end{lstlisting}

\begin{center}
    \color{red}
    EXPLIQUER CE QUE FAIT CETTE COMMANDE \\
    Aussi dire en quoi on s'en servira dans des scripts afin d'accélérer le développement.
\end{center}


\subsubsection{Compilation de Linux depuis le code source}\label{sec:compilation-linux-source}

Ou est le code source linux ?

Comment changer de version du code cloné. 

comment modifier quels modules sont chargés lors de la compilation de linux, interface graphique <-> fichier de config
\subsubsection{Compilation croisée}

Toolchain : qu'est ce que c'est

Variables d'environement

Réalisation de scripts linux pour accélérer le développement

Copy de l'image du noyau pour faire encore plus rapide

